\FloatBarrier
\section{NEAREST} \label{sec:NEAREST}
When we look at the overall error estimation for the method of regularised stokeslets we find the total error to be $\mathcal{O}(\epsilon) + \mathcal{O}(h) + \mathcal{O}(h^2\epsilon^{-1}) + \mathcal{O}(P\epsilon^{-1/P} h^{1-P})$. In order to reduce the overall error, we must reduce $\epsilon$ such that the regularisation error decreases, however doing this necessitates the reduction of $h$ the average quadrature spacing to keep the discretisation error at a similar size. In order to still fully cover our boundary, we must therefore increase the number of quadrature points used in the calculation, which will increase the computation needed to both perform the direct vector-matrix product or the matrix inversion required for resistance problems. 

Methods such as the Boundary Element Method (BEM) \cite{Smith2009AFlow} attempt to fix this problem by decoupling the quadrature and traction discretisation, allowing for the stokeslet kernel to be evaluated by the most appropriate means without affecting the overall number of degrees of freedom. This work has allowed for larger systems to be explored, such as model cilia-driven flows \cite{Sampaio2014Left-rightLaterality,Smith2012SymmetryEmbryo}. Computational experiments show that BEM methods can take up to 3 orders of magnitude less processing time and memory \cite{Smith2009AFlow} when compared to the Nyström method.

The NEAREST method proposed by Smith \cite{Smith2018AEquation} aims to decouple the quadrature and traction discretisation while still preserving the simplicity of the low order quadrature rule used in \cref{eq:Nystrom}. When analysing the values for both the kernel \{$S^\epsilon_{ij}(\bm{y},\bm{x})$\} and force values $\{\bm{f}(\bm{x})\}$ obtained in the method of regularised Stokeslets we find that in most cases the kernel changes much more rapidly than over the surface of the sphere. 
\subsection{Derivation of the NEAREST discretisation}
In order to capture these changes with the smallest number of degrees of freedom necessary, we propose to use a finer quadrature rule with more points so as to capture the rapid changes in the kernel and also a coarser discretisation to capture the slower changes in $\bm{f}(\bm{x})$. In order to achieve this goal we will consider two surface discretisation’s of $\partial D$, one coarse and one fine denoted by $\{\bm{x}_1, \dots, \bm{x}_n\}$ and $\{\bm{X}_1, \dots, \bm{X}_Q\}$ respectively and it is assumed that $N \leq Q$. The coarse discretisation will be our set of points in which we will discretise the force and our fine discretisation will serve as our set of quadrature points for the kernel. In order to map our fine quadrature discretisation onto our set of coarse force points [does not make sense?] will then use nearest-neighbour interpolation defined by $\mathcal{N}:\{ 1, \dots, Q\} \to \{ 1, \dots, N\}$ where 
\begin{equation}
    \mathcal{N}(q) := \underset{n=1, \ldots, N}{\operatorname{argmin}}|\boldsymbol{x}_n-\boldsymbol{X}_q|
\end{equation}
This allows us to define \cref{eq:BIE} as 
\begin{equation}
\begin{aligned}
\label{eq:BIENearest1}
        u_i(\bm{y}) &= \frac{1}{8 \pi \mu} \int_{\partial D} S_{i j}^{\epsilon}\left(\bm{y}, \bm{X}_q\right) f_{j}(\bm{X}_q) d s(\bm{X}_q) \\
        &= \frac{1}{8 \pi \mu} \int_{\partial D} S_{i j}^{\epsilon}\left(\bm{y}, \bm{X}_q\right) f_{j}(\mathcal{N}(q)) d s(\mathcal{N}(q)) \\
        & = \frac{1}{8 \pi \mu} \sum_{q=1}^Q S_{i j}^{\epsilon}\left(\bm{y}, \bm{X}_q\right){f_{j}}[\mathcal{N}(q)]A[\mathcal{N}(q)]
\end{aligned}
\end{equation}
where the final line is the discretisation of the integral with the fine quadrature points. In order to be able to use this interpolation, we need to express the operator $\mathcal{N}(q)$ in its matrix form, a $Q \times N$ sparse matrix defined by
\begin{equation}
\label{eq:NNMatrix}
    \nu [q, \hat{n}]= \begin{cases} 1 & \text { if } \hat{n}=\underset{n=1, \ldots, N}{\operatorname{argmin}}|x_n-X_q| , \\ 0 & \text { otherwise. }\end{cases}
\end{equation}
This provides a direct mapping for each point from our fine quadrature set to the coarse force set. When considering large quadrature sets, we can find cases where a fine quadrature point is equidistant to two or more force points. Under the current form of $\nu [q, \hat{n}]$ the quadrature point is mapped to the first force point in the set. This means that the same set of force and quadrature points can have different results depending on how both sets are arranged. If we instead weight the quadrature point evenly between all equidistant force points, we can eliminate this problem. This can be implemented by replacing the first condition when $\hat{n}=\underset{n=1, \ldots, N}{\operatorname{argmin}}|x_n-X_q|$ with $1/k$ where $k$ is the number of points in $\{x_n\}$ which are equidistant to $\{X_q\}$. This allows a fine point to be mapped to multiple coarse points if they are equidistant, with its contribution evenly distributed between them. The sparse matrix form of $\mathcal{N}(q)$ allows us to express \cref{eq:BIENearest1} as 
\begin{equation}
    u_i(\bm{y}) = \frac{1}{8 \pi \mu} \sum_{n=1}^q  f_{j}(\bm{x}_n) A_n \sum_{q=1}^{Q}S_{i j}^{\epsilon}\left(\bm{y}, \bm{X}_q\right) \nu[q,n] 
\end{equation}
As we considered in the direct product \cref{eq:matrixvectorproduct} we can calculate the velocity at a set of collocation points $\{x_n\}$ with $n=1,\dots,N$ as a matrix-vector product. If we write $\underline{U}$ and $\underline{F}$ as before, the stokeslet matrix $A$ now takes the form of a $3N \times 3Q$ matrix. The nearest neighbour matrix $\nu [q, \hat{n}]$ can be used by taking the Kronecker product of $\nu$ with the $3 \times 3$ identity matrix ($\Pi = \nu \otimes \mathds{1}_{3}$) to map each axis at every point. This allows us to form the system
\begin{equation}
    \underline{U} = A \cdot \Pi \cdot \underline{F}.
\end{equation}
The NEAREST method allows for both the forward matrix product to compute the velocity of the fluid as well as solving the resistance problem through $\underline{F} = (A \cdot \Pi)^{-1} \underline{U}$. The NEAREST method is not directly aimed at competing directly with BEM methods but in practice it does offer an easy mesh-free adaptation to the standard Nyström method which allows for the computation of larger-scale systems than that of the standard Direct product.

By introducing the second discretisation we have now changed the error estimations for the method, we keep the regularisation error of $\mathcal{O}(\epsilon)$ and traction discretisation error $\mathcal{O}(h_f)$. The quadrature error now depends on whether the traction points are contained (within distance $O(\epsilon)$ or closer) in the quadrature set. We introduce the distances $h_q$ and $h_f$ to be the length scales associated with the quadrature and force discretisation respectively. In the contained case we obtain a quadrature error of $\mathcal{O}(\epsilon^{-1}h^2_q) + \mathcal{O}(P\epsilon^{-1/P}h^{1-1/P}_q)$ for all integer $P>3$ and for the disjoint case the error decouples from $\epsilon$ to be $\mathcal{O}(h^3_q\delta^{2}) + \mathcal{O}(Ph^{1-2/P}_q)$ where $\delta$ is the closest distance between quadrature and traction points \cite{Gallagher2019SharpEquation}.

\subsection{Comparison of NEAREST to Nyström method}
In order to illustrate the benefits of using the NEAREST discretisation we will consider the resistance problem \cref{sec:resistance} for a sphere of unit radius for which we have an exact solution to compare our method too (\cref{eq:sphereres}). The construction of the Grand resistance matrix involves solving $6$ resistance problems, three for the unit translations velocities along each axis and three for the unit angular velocities. From each resistance problem we find the total force and moment on the body by summing over all forces of the sphere. This gives us an approximation to $\mathcal{R}$, $\mathcal{R}^\epsilon$. The relative error between the two solutions is then found by 
\begin{equation}
    \text { relative error }=\frac{\left\lVert \mathcal{R}-\mathcal{R}^{\epsilon}\right\rVert_{2}}{\lvert\mathcal{R}\rVert_{2}} \text {, }
    \label{eq:RelativeError}
\end{equation}
where ${\lvert\mathcal{R}\rVert_{2}}$ is the 2-norm of the matrix ($\lvert\mathcal{R}\rVert_{2}=\sup_{x \neq 0}\lvert\mathcal{R}x\rVert_{2} /\lvert x\rVert_{2}$).
The sphere is then discredited into two quadrature sets. A coarse force set with approximate spacing of $h$ and a fine quadrature set with spacing of $h/4$. Figure \ref{fig:NEARESTCOMP} illustrates the relative error for range of quadrature spacing and epsilon. 
\begin{figure}[ht!]
 \centering
 \subfloat[Relative error in Grand Residence matrix for the Nyström method]{\includegraphics[width=0.45\textwidth]{Images/NEAREST/Nystrom.pdf}\label{fig:NYSTROM}}
 \subfloat[Relative error in Grand Residence matrix for the disjoint NEAREST method]{\includegraphics[width=0.45\textwidth]{Images/NEAREST/Disjoint.pdf}\label{fig:DIS}}\\
 \subfloat[Relative error in Grand Residence matrix for the contained NEAREST method]{\includegraphics[width=0.45\textwidth]{Images/NEAREST/Intersecting.pdf}\label{fig:JOIN}}
 \caption{Relative error in calculating the grand resistance matrix for the unit sphere using Nyström and NEAREST}
 \label{fig:NEARESTCOMP}
\end{figure}

While both NEAREST method do benefit from a finer quadrature set all computations were achieved in a similar time with at worst NEAREST taking 29\% longer in computation at the finest discretisation when computed on (M2) using gpuArrays from MATLABs Parallel Computing Toolbox \cite{Gallagher2020} and all methods computing in under 370 seconds. The Nyström method achieves under 2\% errors for several pairs of $h$ and $\epsilon$ but with results quickly diverging as the quadrature error is decreased. We see a similar pattern for the contained NEAREST method where the fine quadrature set includes both the refined quadrature points as well as the coarse force points. This is to be expected as the contained NEAREST method retains the dependence on $\epsilon$ from the nearest method but improves upon the error expected for a given $h$. We do notice a large difference in the disjoint nearest method where the dependence on epsilon has been removed almost entirely with all pairs giving errors of less than 20\% and the smallest values of $h$ giving errors of less than 0.05\%. 
