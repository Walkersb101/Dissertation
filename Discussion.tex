\FloatBarrier
\section{Conclusion}
In this project, we considered the use of the regularised stokeslets method for solving Stokes flow problems. The regular stokeslet method is a widely used approach for solving problems in biological fluid dynamics where the fluid can be approximated by stokes flow. The method of regularised stokeslets is a mesh-free method which dramatically reduced the complexity of implementation but increases the overall computational cost over other methods such as BEM. In \cref{sec:Nbody} we considered how the method of regularised stokeslet could be considered as an N-body problem and how fast algorithms such as the Barnes Hut and KIFMM method can be used to accelerate the matrix-vector product. These algorithms reduce the overall of implementing the method of regularised stokes from order $\mathcal{O}(N^2)$ to $\mathcal{O}(N)$ with the KIFMM. 

We demonstrate the efficiency of the KIFMM in evaluating large regularised stokeslet systems when compared to direct product implementation. Based on our current implementation we suggest that for problems of less than 55000 source points GPU acceleration of the direct product should be considered, however, optimised GPU acceleration of the KIFMM method should prove very efficient at evaluating the matrix products. The KIFMM should be able to provide a black box solution for evaluating the vector-matrix operations allowing for non-specialists to quickly adapt preexisting code to consider larger simulations. A more focused analysis of parameter choices would be useful in order to suggest parameter choices as well as error estimates in advanced simulations. In particular understanding the choice of node capacity as in later experiments we found it plays a large in the overall computational time. More specifically for the case of regularised stokeslets more work needs to be considered on the effect of the choice of both the cutoff function and regularisation parameter on the computation time and error as a result of the KIFMM method.

We considered the use of the KIFMM method in solving more advanced problems in stokes flow where a linear system of equations needs to be solved. While implementation through the direct product allows for standard methods for solving linear systems such as back substitution and QR decomposition, the single direction of the KIFMM method forces the use of the GMRES algorithm to compute the solution to the linear system. When solving resistance problems the linear systems proved relatively easy for GMRES to converge with the overall computation time of the KIFMM beating standard CPU implementations even at small system sizes. When considering the swimming problems for both rods in shear flow and bottom-heavy squirmers the condition of the linear system dramatically increases requiring significantly more iterations than resistance problems with a similar number of degrees of freedom. These systems require the use of a preconditioner to bring both the number of iterations and computation time down to usable levels. In order to compute these systems effectively, we used preconditioners which only considered a small number of interactions present within the system allowing them to be both memory and computationally efficient. A more optimal preconditioner which can consider a larger number of interactions and more accurately represent the inverse stokeslet kernel might also need to be considered through methods such as Inverse FMM algorithms \cite{Ambikasaran2014TheMethod,Coulier2017TheSystems,Alleon1997SparseElectromagnetics}. This would be particularly helpful in the disjoint nearest case where our current preconditioner proves ineffective at improving the convergence of the linear system and as such a better preconditioner would be needed to solve systems of this form. 

While the preconditioned system computes significant faster than the non preconditioned system in most cases when compared to numerical results obtained by Gallagher \cite{Gallagher2020}, GPU implantation of the direct product method using QR decomposition to solve the linear system performers significantly faster. While it is unlikely that the use of KIFMM with GMRES to solve linear systems will be quicker than QR decomposition for smaller-scale systems, a more optimised version implementation of the KIFMM for solving these problems needs to be considered. Profiling of our current implementation reports approximately 40\% of the overall computation is computing the stokeslet kernels which remain the same for each iteration. A more optimised implementation of the KIFMM method where  the kernel matrices, where the multipole to local translations (M2L) are precomputed and stored would reduce the overall computation time for solving linear systems. The kernels for source to local (S2L), local to local (L2L) and (M2M) translations are often computed on each iteration due to the smaller number of translations. In future implementations of the hybrid NEAREST KIFMM precalculation of the S2L matrices could also be considered along with SVD based compression to reduce the memory cost of precalculation \cite{Drineas2006FAST,Cao2012ABEM} as well as dual tree traversal \cite{Yokota2013AnArchitectures:,Dehnen2002AAlgorithm,Carrier2006ASimulations,Wilson2021ATraversal} to optimise the number of kernel operations needed.

Overall the KIFMM and hybrid KIFMM NEAREST method provide a fast and optimal method for evaluating vector-matrix products under the framework of regularised stokeslet with particles number orders of magnitude above current methods. It would be interesting to see how the BEM algorithm would perform when accelerated with the KIFMM method \cite{Cao2012ABEM,Cao2015AAnalysis,Betcke2021Bempp-cl:Library.} to increase the efficiency of solving the problems proposed here. However, in the spirit of regularised stokeslets we note with the future work described above the KIFMM and hybrid KIFMM method could provide a black box algorithm which will allow for non-specialists to compute large stokes flow systems without the need for expertise in numerical methods. solve