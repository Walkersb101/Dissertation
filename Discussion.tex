\FloatBarrier
\section{Conclusion}
In this project, we reviewed and extended high performance implementations for the method of regularised stokeslets for solving Stokes flow problems. The regular stokeslet method is a widely used approach for solving problems in biological fluid dynamics where the fluid can be approximated by Stokes flow. The method of regularised stokeslets is a mesh-free method which dramatically reduces the complexity of implementation, but increases the overall computational cost over other methods such as BEM. In \cref{sec:Nbody} we considered how the method of regularised stokeslet could be considered as an $N$-body problem and how fast algorithms such as the Barnes Hut and KIFMM method can be used to accelerate the computation of the matrix-vector product. These algorithms reduce the overall cost of implementing the method of regularised stokeslets from order $\mathcal{O}(N^2)$ to $\mathcal{O}(N)$ with KIFMM. 

We demonstrated the efficiency of the KIFMM in evaluating large regularised stokeslet systems when compared to direct product implementation. Based on our current implementation, we suggest that for problems of less than 55000 source points, GPU acceleration of the direct product should be considered. A more optimised implementation of KIFMM will provide a black box solution for evaluating the vector-matrix operations allowing for non-specialists to quickly adapt pre-existing code to consider larger simulations. A more focused analysis of parameter choices would be useful in order to suggest optimised parameters as well as error estimates for a range of problems. In particular, understanding the choice of node capacity as in later experiments, we found it plays a large part in the overall computational time. More specifically for the case of regularised stokeslets, more work needs to be considered on the effect of the choice of both the cutoff function and regularisation parameter on the computation time and error as a result of the KIFMM method.

In section \ref{sec:NumericalSims} we considered the use of the KIFMM method in solving more advanced problems in Stokes flow where a linear system of equations has to be solved. While implementation through forming the whole interaction matrix allows for standard methods for solving linear systems, such as back substitution and QR decomposition, the single direction of the KIFMM method forces the use of the GMRES algorithm to compute the solution to the linear system. GMRES converged rapidly for resistance problems, with the overall computation time of the KIFMM method beating standard CPU implementations even at small system sizes. However, due to a dramatic increase in the condition number, GMRES converged much more slowly for both swimming problems considered, requiring significantly more iterations than resistance problems with a similar number of degrees of freedom. These systems require the use of a preconditioner to bring both the number of iterations and the computation time down to usable levels. In order to compute these systems effectively, we used preconditioners which only considered a small number of interactions present within the system, allowing them to be both memory and computationally efficient. 

The speed up obtained though both initial guess and the least square commutator is clearly illustrated in \cref{tab:Preconditioning} where we see dramatic decrease in the computation time when using both methods. While both methods aim to decrease overall computation time in two different ways using both methods together will not decrease the overall computation time much if at all due to the reduced number of iterations in the preconditioned systems. We did aim to include comparisons with current implementations using QR decomposition, however due to the number of quadrature points considered, the memory requirements were too large for our available resources at this time. We could have considered a comparison using NEAREST on (M1) as block based construction of the kernel matrix would be possible with available hardware however due to the need for a better preconditioner for disjoint force and quadrature sets we will leave this up for further work. Methods such as Inverse FMM \cite{Ambikasaran2014TheMethod,Coulier2017TheSystems,Alleon1997SparseElectromagnetics} would allow for more acurate representation of the inverse stokeslet kernel and allow for closer representation of the LU decomposition needed for optimal GMRES convergence. These methods may be able to reduce the computation time down even further and allow for systems of this scale to be used with methods such as \textit{ODE45} to see the time evolution of these problems in reasonable computational time frames. 

The KIFMM NEAREST hybrid method does prove promising at allowing for the computation of larger systems decreasing the overall computation time for the 2016 swimmer case by 33\% for the joint NEAREST discretisation while only slightly increasing the expected error in the motion of the swimmers as see in \cref{sec:NEAREST,sec:Hybrid}. In our simulation of bottom-heavy squirmers we used the disjoint NEAREST discretisation such that we could consider the interactions of a larger group of squirmers. The velocity slices shown in \cref{fig:SquiremerGyroPo} only considered up to $t=5.18$ as the upper boundary fails to contain the squirmers in the plotted domain. To constrain the squirmers in future simulations we need to consider either increasing the resolution of both discretisation, using the method of images \cite{Ainley2008TheStokeslets} or replacing the regularised stokelet kernel with a regularised blakelet kernel \cite{blake_1971,Cortez2015}. Both increasing the resolution of our discretisation and using the method of images would require increasing the number of stokeslet points needed to be evaluated further increasing the computation time. While the blakelet solutions should just be a change in our kernel function there has been no analysis done on how it interacts with the KIFMM method and as such should be considered for further work.    

While the preconditioned systems compute significantly faster than the non preconditioned system, in most cases, when compared to numerical results obtained by Gallagher \cite{Gallagher2020}, GPU implementation of the direct product method using QR decomposition solves the linear systems significantly faster. While it is unlikely that the use of KIFMM with GMRES to solve linear systems will be quicker than QR decomposition for smaller-scale systems, a more optimised implementation of the KIFMM for solving these problems needs to be considered. Profiling of our current implementation reports approximately 40\% of the overall computation is computing the stokeslet kernels which remain the same for each iteration. A more optimised implementation of the KIFMM method where the S2L and M2L translations are precomputed and stored would reduce the overall computation time for solving linear systems. In further work a change in programming language would also worth considering to further optimise the parallelisation of the KIFMM method. 

Overall the KIFMM and hybrid KIFMM NEAREST methods provide a fast and optimal technique for evaluating vector-matrix products under the framework of regularised stokeslets, with the number of force points at orders of magnitude above current methods. It would be interesting to see how the BEM algorithm would perform when accelerated with the KIFMM method \cite{Cao2012ABEM,Cao2015AAnalysis,Betcke2021Bempp-cl:Library.} to increase the efficiency of solving the problems proposed here. However, in the spirit of regularised stokeslets, we note with the future work described above, the KIFMM and hybrid KIFMM method could provide a black box algorithm which will allow for non-specialists to compute large Stokes flow systems without the need for expertise in numerical methods.