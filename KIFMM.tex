\section{Kernel-Independent Fast Multipole Method}
If we consider a many body problem formed of $N$ source/quadrature points $\{\mathbf{X_{0,n}}\}$ each with an associated potential/Force $\{\mathbf{f_{0,n}}\}$ that we will evaluate on $M$ collocation points $\{\mathbf{x_m}\}$ then we can compute the velocity at each $\mathbf{x_m}$ $m=1,2,\dots,M$ through
\begin{equation*}
    \mathbf{u}(\mathbf{x_j}) = \sum_{i=1}^N \Phi(\mathbf{x_{0,j}},\mathbf{x_i})\mathbf{f_{0,i}}
\end{equation*}
where $\Phi$ is the green's function associated with the underlying partial differential equation governing the dynamics of the system. In particular $\Phi$ is $S^\epsilon$ for our kernel for the regularised Stokeslet kernel where we compute the velocity at as set of collocation points based on a set of body forces defined on solid boundaries within the fluid. This can we written as a matrix-vector product as described in \cref{eq:matrixvectorproduct}. This thas a computational complexity of $\mathcal{O}(NM)$ which for either a large number of collocation points or quadrature points quickly becomes prohibitive. In order to compute large systems we need to look at faster methods to approximate this vector matrix product, the fast multipole method (FMM) can approximate \cref{eq:matrixvectorproduct} with complexity $\mathcal{O}(N+M)$. The FMM algorithm uses Taylor expansion of the kernel about the quadrature points in order to approximate long range interaction between distant quadrature and target points \cite{Beatson1997AMethods,Tornberg2008,Wang2007AEquations,Yokota}. This method proves time consuming to find for particular kernels such for regularised Stokeslets, it also means that the entire program is kernel dependent and any changes to the kernel needs a complete recalculation of the method.
We instead use a new algorithm based on the ideas of FMM to create a kernel independent fast multipole algorithm (KIFMM) \cite{Ying2004, Ying2005,Rostami2016Kernel-independentStokeslets,Yan}. Using equivalent potentials instead of Taylor expansions we can approximate long range interactions using only kernel evaluations without computing every particle to particle interaction. This dramatically reduces the number of kernel evaluations verses the direct solution. 
\subsection{Hierarchical decomposition of the computational domain}
Both FMM and KIFMM are based of the Hierarchical decomposition of the computational domain. We let $\mathcal{D}$ be the computational domain, which we define to be a cube in $\mathcal{R}^3$ such that it encompasses all points in $\{\mathbf{x_{0,n}}\}$ and $\{\mathbf{x_m}\}$. We typically definite this to be the smallest possible cube which includes all the points. The FMM method builds a Octree structure of cubes in 3D with the cube $\mathcal{D}$ as the root node.  
\subsection{Equivalent surfaces/coronas}
\subsection{Numerical approximation of the force densities}
\subsection{Evaluation}
\subsection{Comparison of KIFMM method vs Direct solver}
