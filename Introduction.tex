\section{Introduction}
When considering the physical behaviour of microorganisms, the dynamics of the surrounding fluid are important, especially in the interaction of multiple organisms. Experiments have shown that at high enough density, the collective motion of cells can form unique and complex arrangements which define and explain their evolutionary traits \cite{Moore2002ExceptionalMouse,Cisneros2011DynamicsConcentration,Piddini2005Biophysics:Cells}. While experiments on microorganisms provide an understanding of the biological systems, mathematical models allow for a deeper understanding of the underlying physics. 

The biological relevance of flows at this scale has motivated over 150 years of research into understanding such fluids. Originally studied by Stokes \cite{Stokes2010OnPendulums}, the steady-state Stokes equations (eq.~\ref{eq:StokesFlow}) govern flows at this scale, where viscous forces dominate over the inertial forces. While analytical methods have been used to make significant progress in understanding Stokes flow, in particular the analysis of flows around sperm-like swimmers \cite{Hancock1953TheLiquids,GRAY1955TheSpermatozoa,Taylor1951AnalysisOrganisms}; they are unable to provide results for more complex and relevant problems which typically involve multiple swimmers and large amplitude motion. Most recent work on Stokes flows has been focused on methods to computationally solve the steady-state Stokes equations \cite{Biros2003,Rostami2016Kernel-independentStokeslets,Rostami2019FastBiofluids,Wang2006AlgorithmsSimulation,Sierou2001AcceleratedSimulations,Selmi2007FastComplexity,Smith2018AEquation,Gallagher2020}. 

As the Stokes equations are linear and elliptical partial differential equations \cite{Pozrikidis1992BoundaryFlow}, they can be formulated in terms of a boundary integral equation \cite{Pozrikidis2002ABEMLIB,Acrivos1975StokesSolution,Pozrikidis1992BoundaryFlow,Tran-Cong1987APropulsion,Stakgold1968Boundary2}; whereas typical methods to solve differential equations such as the finite element method \cite{Wagner2001,Kouhia1995} would require discretisation over the whole volume which is computationally expensive, particularly given the infinite fluid domain typically considered, we will be considering the Stokes equations which can be solved as an integral over the boundary. Many numerical methods have been derived to solve the boundary integral equations, including a boundary element method (BEM) for Stokes flow developed by Phan-Thien et al \cite{Tran-Cong1987APropulsion}, and have been used to model these systems. While these methods are both accurate and efficient they present two challenges, particularly for researchers who are not computational specialists. The first challenge is the need to generate a smooth surface mesh on the boundary of the object. While this is relatively easy for simple geometry it is not often obvious how these meshes can be formed for more complex moving geometry. While automatic mesh generators \cite{Pozrikidis2002ABEMLIB,gmsh,schroeder2006visualization} have allowed for simpler implementation of BEM methods they do not provide a robust and efficient solution to all geometries and often lead to unnecessary calculations where the mesh has been refined beyond that which is required \cite{gmsh,schroeder2006visualization}. The second problem arises from the singular solutions (eq.~\ref{eq:singularsolutions}) which have the requirement for semi-analytical quadrature methods or specialised numerical qudaratures.

In order to address both these issues, Cortez et al \cite{Cortez2001,Cortez2005} introduced the method of regularised stokeslets, which both removes the need for a connected mesh and the singularities in the solutions. The core idea is the replacement of the force evaluated at a singular point with spatially-smoothed force localised at the same point. We review the method of regularised stokeslets in \cref{sec:MRS} including the derivation of its boundary integral equation. While this method proves very easy to implement it does come at a computational cost, due to the surface traction varying much slower than the stokeslet kernel. Therefore the regularised stokeslet method requires significantly more degrees of freedom, corresponding to the traction discretisation, than typical BEM implementations to achieve similar levels of accuracy. This is further compounded by the coupling of the discretisation spacing with the regularisation parameter which as shown in \cref{sec:MRS} furthers the need to refine the surface discretisation. 

In chapters \cref{sec:Nbody,sec:KIFMM} we will consider how we can consider solving Stokes flow using the method of regularised stokeslets as a $N$-body problem and how it is possible to reduce the traditional $\mathcal{O}(N^2)$ complexity of the problem to $\mathcal{O}(N)$ though the use of fast summation methods, in particular the Kernel Independent Fast Multipole Method (KIFMM) developed by Ying, Biros and Zorin \cite{Ying2004,Rostami2016Kernel-independentStokeslets,Rostami2019FastBiofluids} which we review in detail in \cref{sec:KIFMM}. We will also consider an alternative `nearest neighbour' discretisation (NEAREST) by Smith \cite{Smith2018AEquation,Gallagher2020,Gallagher2018MeshfreeCells} in \cref{sec:NEAREST} which aims to retain the mesh-free nature  of the standard implementation while dramatically reducing the computational cost of the method by separating the traction discretisation and numerical quadrature. The main contribution of this paper is the formulation of a hybrid method which uses both nearest neighbour interpolation to reduce the overall number of degrees of freedom and the KIFMM algorithm to quickly evaluate the particle to particle interactions. Finally, \cref{sec:NumericalSims} explores how KIFMM and the hybrid KIFMM NEAREST methods can be used to solve problems in Stokes flow, in particular swimming problems, where the motion of multiple swimmers can be found from the swimmers boundary deformation and force/moment balance. The single direction of FMM algorithms promotes the use of Generalized Minimal RESidual method (GMRES) and an appropriate preconditioner based on work by Rostami and Olson \cite{Rostami2019FastBiofluids} to solve the linear systems.