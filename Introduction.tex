\section{Introduction}
At the scale of microorganism fluid dynamics plays an important and defining role in the evolution and behaviour of these cells particularly in the interactions of multiple cells. Experiments have shown that at high enough density the collective motion of cells can form unique and complex arrangements which define and explain their evolutionary traits. While experiments on microorganism provide an understanding into the larger biological systems, mathematical models allows of a deeper understanding of the underlying physics. 

The biological relevance of flows at this scale has motivated over 150 years of research into understanding fluids at this scale. Originally studied by Stokes \cite{Stokes2010OnPendulums}, the steady-state Stokes equations \cref{eq:StokesFlow} govern flows at this scale where viscous forces dominate over the inertial forces. While analytical analysis has been used to make significant progress into understand Stokes flow in particular the analysis of flows around sperm like swimmers \cite{Hancock1953TheLiquids,GRAY1955TheSpermatozoa,Taylor1951AnalysisOrganisms}. However although the analysis done analytically provides significant progress into understanding biological flows they are unable to provide results to more complex and relevant problems which typically involve multiple swimmers and large amplitude motion. Most recent work on stokes flows has been focused on methods to computationally solve the steady state Stokes equations. 

As the Stokes equations are linear and elliptical partial differential equations they can be formulated in terms of a boundary integral equation; where typical method to solve differently equation such as finite element method would require the discretisation over the whole volume, particularly given the infinite volumes we will be concerning, the equation can be solve as a integral over the boundary. This reduction dimension vastly reduces the size of the problem which is obtained from the discretisation of the problem and removes the need to mesh the whole volume. Many numerical method have been derived to solve the boundary integral equations, including a boundary element method (BEM) for stokes flow developed by Phan-Thien et al \cite{Tran-Cong1987APropulsion}, have been use to model these systems. While these methods are both accurate and efficient they present two challenges particularly for researchers who are not computational specialists. The first challenge is the need to generate a smooth surface mesh on the boundary of the object. While this is relatively easy for simple geometry it is not often obvious how these meshes can be formed for more complex moving geometry. While automatic mesh generators have allowed for simpler implantation of BEM methods they do not provide a robust and efficient solution to all geometries and often lead to unnecessary calculations where the mesh has been refined  beyond what is necessary. The second problem arises from the singular solution (\cref{eq:singularsolutions}) which arise from the is the requirement for semi-analytical quadrature method which are hard to implement effectively. While libraries such as BEMLIB \cite{BEMLIB} have mostly eliminated this problem for the end user it does add a layer of complexity to the calculations.

In order to address both these problems Cortez et al \cite{Cortez2001,Cortez2005} introduced the method of regularised stokeslet which both removes the need for a connected mesh and removes the singularities in the solutions. The core idea is the replacement of the stokes flow formed by a Dirac delta force-per-unit-volume distribution, with a flow formed by cutoff function. We review this method in \cref{sec:MRS} including the derivation of its boundary integral equation. While this method proves very easy to implement it does come at a computational cost, where the number of degrees of freedom needed to discrete the boundary is significantly higher than typical BEM implementations in order to achieve similar levels of accuracy. This is further compounded by the coupling of the discretisation spacing with the regularisation parameter which as shown in \cref{sec:MRS} furthers the need refine the surface discretisation. 

In chapters \cref{sec:Nbody,sec:KIFMM} we will consider how we can consider solving Stokes flow using the method of regularised stokeslets as a $N$-body problem and how it is possible to reduce the traditional $\mathcal{O}(N^2)$ complexity of the problem to $\mathcal{O}(N)$ though the use of Fast summation methods, in particular the Kernel Independent Fast Multipole Method (KIFMM) developed by Ying, Biros and Zorin \cite{Ying2004} which we review in detail in \cref{sec:KIFMM}. We will also consider an alternative `nearest neighbour' discretisation by Smith \cite{Smith2018AEquation} in \cref{sec:NEAREST} which aims to retain the meshfree nature  of the standard implementation while dramatically reducing the computational cost of the method. The main contribution of this paper is the formulation of a hybrid method which uses both nearest neighbour interpolation to reduce the overall number of degrees of freedom and the KIFMM algorithm to quickly evaluate the particle to particle interactions. Finally \cref{sec:NumericalSims} explores how KIFMM and the hybrid KIFMM NEAREST methods can be use to solve problems in stokes flow, in particular swimming problems the motion of multiple swimmers can found from there from given given boundary deformation and force/moment balance. This promotes the use of GMRES and an appropriate preconditioner based on work by Rostami and Olson \cite{Rostami2019FastBiofluids}.

